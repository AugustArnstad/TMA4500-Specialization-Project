Statistical inference about the parameters of a model is a fundamental part of statistical analysis. 
In many applications it is desirable to understand how important each covariate is in explaining the observed response variable.
By specifying the importance of each variable, researchers gain a better understanding of the complex relationships in the statistical model, thereby improving their knowledge and quality of research.
Currently, there are several methods that try to deduce the importance of variables in a statistical model.
These methods include testing a null hypothesis by using parameter estimates, confidence intervals or $p$-values
This thesis aims to provide a method that decomposes the model variance in a Bayesian framework, to emphasize statistical inference rather than threshold-based, for example regarding a covariate as significant if $p<0.05$, interpretations.
\newline
\newline 
We introduce a Bayesian measure of variable importance, which we call the \textit{Bayesian Variable Importance} (BVI) method. 
The BVI method adapts the key aspects of established methods for decomposing the variance of a random intercept model, into the Bayesian framework.
To do this, we use the relative weights method as our starting point.
The relative weights method, which can be considered an approximation of more computationally exhausting methods, projects correlated covariates into an uncorrelated space.
In the respective space the Bayesian model is fit, and the results are back-transformed to the original covariate space. 
The results include posterior distributions for each model covariate and the calculated variance that each covariate contributes to the full model variance. 
Furthermore, the package \texttt{BayesianImportance} is developed to implement the Bayesian Variable Importance method in the statistical software R.
\newline 
\newline 
From a simulation study it is shown that the Bayesian Variable Importance method provides a computationally fast and proper decomposition of the model variance.
Analysis on a single, simulated dataset is shown, highlighting the statistical inference one can obtain from the BVI method.
The results, such as posterior distributions, are discussed in further detail and further work is outlined.
It is discussed that there are many possibilities with the BVI method going forward and that relative importance in a Bayesian framework is an area where much work can be done in the future. 


