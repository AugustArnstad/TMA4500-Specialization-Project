In this thesis we aimed to provide a novel Bayesian variable importance measure, the BVI method, for the random intercept model. 
The BVI method projects the fixed covariates into an uncorrelated space to easy computational efforts. 
These uncorrelated covariates are used when the model is fit with a Bayesian approach, where we relied upon the INLA framework.
The fitted model can be used to draw samples from the posterior distribution of the model parameters, which are back-transformed to the original covariate space which is of interest.
It is in this original covariance space inference is made.
\newline
\newline
From a simulation study we have shown that the BVI method provides results that suggest it provides a proper decomposition and performs well when compared to more established methods in the frequentist framework.
Further, the BVI method allows for extensive inference through posterior distributions and is implemented in the package \texttt{BayesianImportance} in the statistical software R.
We lastly outline some desired improvements and extensions of the BVI method, before a Bayesian relative importance framework for the generalized linear mixed models can be accomplished.
Hopefully, the Bayesian Variable Importance method is the first step towards this goal.