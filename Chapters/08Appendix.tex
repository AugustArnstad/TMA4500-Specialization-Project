

\chapter*{A - GitHub repository}
\addcontentsline{toc}{chapter}{\protect\numberline{}A - Github repository} 

All code used to produce results and all latex files in this document are included in the Github repositories linked below. Further explanations are given in the readme-files. 
\subsection*{GitHub repository link}
\begin{itemize}
    \item \url{https://github.com/AugustArnstad/BayesianImportance}
    \item I will add the latex folder of the project before handing it in.
\end{itemize}




%%%%%%%%%%%%%%%%%%%%%%%%%%%%%%%%%%%%%%%%%%%%%%%%%%%%%%%%


\chapter*{B - Bayesian Variable Importance usage}
\addcontentsline{toc}{chapter}{\protect\numberline{}B - Bayesian Variable Importance usage} 

In this section, an example R code snippet is provided for demonstration purposes. This code creates a random intercept model according to the Bayesian Variable Importance method described in the thesis, and includes some relevant plots and shows the main capabilities of the package. 

\section*{R Code Example}
\begin{lstlisting}[language=R, caption=Usage of the Bayesian Importance package with plots and examples.]
  ## INSTALLING THE PACKAGE
  # This section ensures the devtools package is installed, which is required for installing packages from GitHub. We then install the BayesianImportance package directly from GitHub using devtools::install_github(). In the package under the Hello.R file, all functions are defined with corresponding documentation.
  ```{r}
  # If not already installed, install the 'devtools' package
  if(!require(devtools)) install.packages("devtools")
  
  # Install BayesianImportance
  devtools::install_github("AugustArnstad/BayesianImportance")
  
  ```
  
  ## SIMULATE DATA
  # In this part, we simulate data to demonstrate the functionality of the BayesianImportance package. We generate random variables with different correlation structures, random effects, and an error term. The data is then structured into data frames for further analysis. If you have a suitable dataset you can use this instead.
  
  ```{r}
  library(BayesianImportance)
  library(INLA)
  library(mnormt)
  library(ggplot2)
  library(reshape2)
  library(RColorBrewer)
  set.seed(1234)
  
  n <- 10000
  nclass_gamma <- 200
  nclass_eta <- 100
  
  mu <- c(1,2,3)
  
  sigma <- matrix(c(1, 0.3, 0.5, 0.3, 1, 0.4, 0.5, 0.4, 1), 3, 3)
  
  #Sample a standardized correlated design matrix
  X <- rmnorm(n, mu, sigma)
  
  #Add random effects
  gamma <- rep(rnorm(nclass_gamma, 0, sqrt(1)), each=n/nclass_gamma)
  eta <- rep(rnorm(nclass_eta, 0, sqrt(1/2)), each=n/nclass_eta)
  epsilon = rnorm(n, mean=0, sd=sqrt(1))
  beta <- c(1, sqrt(2), sqrt(3))
  
  #Define some formula
  Y1<-  1 + beta[1]*X[, 1] + beta[2]*X[, 2] + beta[3]*X[, 3] + gamma + epsilon # + eta
  Y2<-  1 + beta[1]*X[, 1] + beta[2]*X[, 2] + beta[3]*X[, 3] + gamma + eta + epsilon
  
  #Collect as a dataframe
  data_bayes1 = data.frame(cbind(Y1, X = X))
  data_bayes1 = data.frame(cbind(data_bayes1, gamma=gamma)) 
  
  data_bayes2 = data.frame(cbind(Y2, X = X))
  data_bayes2 = data.frame(cbind(data_bayes2, gamma=gamma)) 
  data_bayes2 = data.frame(cbind(data_bayes2, eta=eta))
  
  names(data_bayes2)
  ```
  
  
  ## USAGE
  # Here we demonstrate the usage of the BayesianImportance package. We fit two Bayesian models and sample posterior distributions for different simulated datasets using functions from the package BayesiannImportance. We fit one model with a single random intercept and one model with two random intercepts
  ```{r}
  set.seed(1234)
  
  model1 <- run_bayesian_imp(Y1 ~ V2 + V3 + V4 + (1 | gamma), data=data_bayes1)
  posteriors1 <- sample_posteriors(Y1 ~ V2 + V3 + V4 + (1 | gamma), data=data_bayes1, 5000, n)
  
  model2 <- run_bayesian_imp(Y2 ~ V2 + V3 + V4 + (1 | gamma) + (1 | eta), data=data_bayes2)
  posteriors2 <- sample_posteriors(Y2 ~ V2 + V3 + V4 + (1 | gamma) + (1 | eta), data=data_bayes2, 5000, n)
  ```
  
  ```{r}
  variance_marginals_list <- lapply(model1$marginals.hyperpar, function(x) inla.tmarginal(function(t) 1/t, x))
  
  res <- variance_marginals_list$`Precision for the Gaussian observations`
  
  inla.mmarginal(res)
  ```
  
  
  ## PLOTTING
  # This code block defines a function plot_posterior that visualizes posterior distributions and relative importance from the sampled posteriors. The function creates density plots and overlays line plots for variance marginals. This can be modified for the specific problem at hand, and is not included in the package since it is a problem specific function.
  ```{r}
  plot_posterior <- function(betas, importances, marginals, importance=FALSE, theoretical=FALSE, eta=FALSE) {
    
    if (importance) {
      mat_long <- melt(as.data.frame(importances))
    } else {
      mat_long <- melt(as.data.frame(betas))
    }
    names(mat_long) <- c("Variable", "Value")
    
    df_marginals <- do.call(rbind, marginals)
    df_marginals$Effect <- as.factor(df_marginals$Effect)
    
    if (!importance){
      plot_title = paste("Posterior Distributions")
    }else{
      plot_title = paste("Posterior Relative Importance")
    }
    
    if (eta) {
      random_effect_labels <- c(expression(sigma[eta]^2), expression(sigma[alpha]^2), expression(sigma[epsilon]^2))
      num=6
    }
    else{
      random_effect_labels <- c(expression(sigma[alpha]^2), expression(sigma[epsilon]^2))
      num=5
    }
    
    fixed_effect_labels <- if (importance) {
      c(expression(beta[1]^2), expression(beta[2]^2), expression(beta[3]^2))
    } else {
      c(expression(beta[1]), expression(beta[2]), expression(beta[3]))
    }
    colors=rainbow(5)
    
    p <- ggplot() +
      geom_density(data = mat_long, aes(x = Value, fill = as.factor(Variable)), alpha = 0.5) +
      geom_line(data = df_marginals, aes(x = x, y = y, color = Effect)) +
      labs(#title = plot_title, 
           x = "Importance", y = "Density/Frequency") +
      theme_minimal() +
      theme(legend.position = "right", #USE NONE FOR NO LEGEND
            legend.text = element_text(size = 20),
            legend.title = element_text(size = 20),
            plot.title = element_text(size = 20, face = "bold"),
            axis.title.x = element_text(size = 20),
            axis.title.y = element_text(size = 20),
            strip.text = element_text(size = 20),
            axis.text.x = element_text(size = 20),
            axis.text.y = element_text(size = 20)) +
      scale_fill_manual(name = "Fixed effects", 
                        values = colors[1:3], 
                        labels = fixed_effect_labels) +
      scale_color_manual(name = "Random effects", 
                         values = colors[4:num], 
                         labels = random_effect_labels)
      
    if (theoretical) {
      if(eta){
        p <- p + geom_vline(xintercept = 1/9, color = "black", linetype = "dashed") +
              geom_vline(xintercept = 1/18, color = "black", linetype = "dashed") +
              geom_vline(xintercept = 2/9, color = "black", linetype = "dashed") +
              geom_vline(xintercept = 3/9, color = "black", linetype = "dashed")
      }
      else{
        p <- p + geom_vline(xintercept = 0.125, color = "black", linetype = "dashed") +
              geom_vline(xintercept = 0.25, color = "black", linetype = "dashed") +
              geom_vline(xintercept = 0.375, color = "black", linetype = "dashed")
      }
    }
    return(p)
  }
  
  plot1 <- plot_posterior(posteriors1$beta, posteriors1$importance, posteriors1$marginals, importance = TRUE)
  
  plot2 <- plot_posterior(posteriors2$beta, posteriors2$importance, posteriors2$marginals, importance = TRUE, eta=TRUE)
  
  ```
  
  # These plots are generated for each model's posteriors. We create and store the plots in variables plot1, plot2, plot3, and plot4 for the respective models.
  ```{r}
  plot1
  
  plot2
  ```
  
  
  ## GELMAN R2
  # Here, we compute and visualize the Gelman conditional R2 metrics for each model. This metric gives insight into the proportion of variance explained by the fixed effects in the models.
  ```{r}
  colors=rainbow(5)
  # Create a combined plot with different colors for each distribution
  marginal_var <- ggplot() +
    geom_density(data = as.data.frame(posteriors1$r2), aes(x = V1), fill = colors[1], alpha = 0.5) +
    #geom_vline(xintercept = variance_explained$R2_marginal[1], color = colors[1], linetype = "dashed") +
    
    geom_density(data = as.data.frame(posteriors2$r2), aes(x = V1), fill = colors[2], alpha = 0.5) +
    #geom_vline(xintercept = variance_explained$R2_marginal[2], color = colors[2], linetype = "dashed") +
    
    theme_minimal() +
    theme(legend.position = "none",
          axis.title.x = element_text(size = 20),
          axis.title.y = element_text(size = 20),
          axis.text.x = element_text(size = 20),
          axis.text.y = element_text(size = 20)) +
    labs(x = expression(R^2), y = "Frequency")
  
  marginal_var
  ```
  
  ```{r}
  
  # Create a combined plot with different colors for each distribution
  conditional_var <- ggplot() +
    geom_density(data = as.data.frame(posteriors1$r2_cond), aes(x = V1), fill = colors[1], alpha = 0.5) +
    #geom_vline(xintercept = variance_explained$R2_conditional[1], color = colors[1], linetype = "dashed") +
    
    geom_density(data = as.data.frame(posteriors2$r2_cond), aes(x = V1), fill = colors[2], alpha = 0.5) +
    #geom_vline(xintercept = variance_explained$R2_conditional[2], color = colors[2], linetype = "dashed") +
    
    theme_minimal() +
    theme(legend.position = "none",
          axis.title.x = element_text(size = 24),
          axis.title.y = element_text(size = 24),
          axis.text.x = element_text(size = 24),
          axis.text.y = element_text(size = 24)) +
    labs(x = expression(R^2), y = "Frequency")
  
  conditional_var
  ```
\end{lstlisting}




%%%%%%%%%%%%%%%%%%%%%%%%%%%%%%%%%%%%%%%%%%%%%%%%%%%%%%%%